\chapter{Accountability Work: Examining the Values, Technologies and Work Practices that Facilitate Transparency in Charities}
\label{sec:emp01}

\textbf{Note to readers: This is basically a ripped version of a paper that was rejected from the ACM CHI conference. As such it currently refers to itself as a paper and not a chapter. It also contains a background section and methods section, which need to change. It just needs to be in here so that I begin to work on it}

%Paragraph 1: This paper... + the seriousness of the problemspace and what the digital stuff does
This paper discusses the role of data technologies in Charitable Organisations (charities) as they seek to become transparent and accountable in their work and their financial practices. Charities play an important role in society, oftentending to issues of importance to populations and communities where both the private and state sectors have lacked engagement (or resources) \cite{salamon_rise_1994}. Due to the nature of much of their funding, through grants and public donations, charities are required to demonstrate to stakeholders both a commitment to their aims and a competency in their financial practices \cite{macmillan_relationship_2005, oliver_what_2004}. Data technologies play an increasingly important role in mediating transparency for organisations by supporting the online reporting and publishing of financial data \cite{meijer_understanding_2009}, while the production of open data is widely claimed to be synonymous with transparency in dialogues around government and business \cite{coleman_lessons_2013, goldstein_open_2013, gordon_making_2013}. Recent work within HCI has examined the use of open data by charities for constructing narratives \cite{erete_storytelling_2016}, the use of data for metrics for reporting and understanding \cite{elsden_resviz:_2016}, and has provided insight into how digital systems can provide more comprehensive forms of transparency in these organisations \cite{marshall_accountable:_2016}. However, thus far there is little understanding of how technologies like these, and more commonplace data technologies, and data work, integrate into the daily, lived, work of charities.

%; or on how this same work would in turn further affect the design of sociotechnical systems in this space.
% Hmm. I liked the above sentence. Poo.

% Paragraph 2: In order to examine these issues we did this...
Our research sets out to address this deficit and better understand how transparency and accountability manifest in the daily activities of charities. We conducted a qualitative study of work practice in a charity that conducts youth work for economically deprived and migrant communities in the North of England. Over a period of 7 months the first author spent an average of 2 days a week conducting ethnographic fieldwork at the charity's main community hub and office, and participated in both delivery of community-facing activities and administrative work. The fieldwork was oriented towards developing a praxeological account of how financial work is performed within the organisation \cite{crabtree_doing_2012}. The findings from our fieldwork provide insight into the tools and processes used by members of such a setting to organise and make sense of their activities and finances and, more crucially, the work that is required to make this \textit{transparent} and \textit{accountable} to others. We also discuss the tensions that exist between the everyday execution of charitable work and the legal or contractual obligations to account for it in particular ways. In doing so we highlight how organisations can navigate these issues in order to make themselves accountable not only `on paper' but to those who rely on the charity's projects and services.


% Paragraph 3: The contributions of the paper are this
This work contributes to the field of HCI in two ways. First, we produce an account of a charity's work practices as they relate to data technologies, financial transparency, and accountability. In doing this we extend HCI's discussion of technologies and organisational accountability \cite{marshall_accountable:_2016} by illuminating how this relates to the lived practice of those whose work these technologies will impact. Secondly, in discussing the tensions that exist in these settings, we extend current work in HCI around sharing data for sensemaking, human-data interaction and organisational transparency, by presenting a set of design recommendations for future systems. In doing so, we discuss how HCI can affect real change in charities worldwide by facilitating engagement with their work's value, and communicating this to others.

\section{Background and Related Work}
% Charities are awesome
It is generally understood that charities play an important role in society. They perform work in areas and matters generally left unattended by state or private sectors. This includes driving grass-roots development and human services \cite{salamon_rise_1994}, the creation and sustenance of social capital within communities and for particularly marginalised populations. \textit{Social Capital} \cite{field_social_2003, mendel_doing_2014}. It can be said that a charity's very existence indicates a substantial need for its model of service delivery, due to the failure of the market to regulate for-profit entities which may engage in potentially harmful or exploitative practices\cite{hansmann_role_1980}.

% Transparency is awesome blah blah blah generates accountabilities
The importance of their role and their funding mechanisms makes accountability a cornerstone in the public's relationship with Charities. This is for, at least, two reasons. First, due to the important role that charities play in society, making an organisation accountable for actions it takes ensures that it is true to its mission and do not abuse the trust of the public and other stakeholders who might support its cause \cite{frumkin_accountability_2006, jacobson_lifting_2005}. Second, due to the fact that charities are often funded with grants from private or public funds, it is argued that they should be held accountable and act transparently in regards to their financial practices. This is to ensure that they are seen to be using funds both appropriately and efficaciously. Furthermore, due to the nature of charitable funding, this means having multiple and diverse stakeholders to which they must be accountable \cite{krashinsky_stakeholder_1997, macmillan_relationship_2005}.

% Transparency is effectively trying to facilitate interactions with the work of an organisaiton and stakeholders that make them accountable
There are multiple ways in which an organisation can be said to be accountable. This can include: the extent to which its stakeholders can direct its activity \cite{koppell_pathologies_2005}; how it can be called upon to justify its actions \cite{fox_uncertain_2007}; and how it can be made to adhere to responsibilities through legal frameworks \cite{koppell_pathologies_2005}. These theories of accountability have impact on the way that charities conduct their everyday operations regarding work and spending. Accountability shares a complicated relationship with financial transparency; the latter often being cited as means to provide the former \cite{hood_accountability_2010}. Koppell describes transparency as the foundational element for accountability upon which all of the other forms are built \cite{koppell_pathologies_2005}. Fox conceptualises an intersection between the two called `answerability' \cite{fox_uncertain_2007}. For organisations there are many ways to be transparent such as passiveley revealing information or actively engaging stakeholders \cite{oliver_what_2004, schauer_transparency_2011}, or choosing to focus on outcomes or spend \cite{heald_varieties_2006}. The position of this paper is that all forms of transparency share a concern over the dissemination and consumption of information. The purpose of being transparent, therefore, is ultimately to facilitate interactions between an organisation, its work, and stakeholders (such as funders or the wider public). These interactions are what mediate accountability, and these interactions are increasingly facilitated digitally \cite{meijer_understanding_2009, oliver_what_2004}.

% If transparency is ultimately interacting with data, then here's some interactions with data (resviz and storytelling).
The ideals and beleifs surrounding use of data and personal informatics has seen the field of HCI explore the politics and design of systems that facilitate interactions with data. At an organisational level, studies have demonstrated how charities have used open data to form narratives around local conditions \cite{erete_storytelling_2016}, while others have highlighted how visualisations around organisational metrics (including funding) support the use of data for reporting, understanding, and providing insight in highly politicised environments \cite{elsden_resviz:_2016}. In the personal sphere, concepts of data lockers allow external processors to interact with one's data while maintaining personal control \cite{mcauley_dataware_2011} and data itself is likened to a boundary object forming part of the infrastructure of everyday life \cite{crabtree_human_2015}.

% This mirrors modern transaprency efforts, with a lot of data being available online, although with some criticisms (ME). Additionally, transparency is often heralded as being in conflict between effective practice; taking additional effort to produce an account of work (financial or otherwise).
Modern transparency efforts often legally stipulate that charities and other NPOs submit documentation for auditing and subsequent public consumption. Examples include the US' Internal Revenue Service (IRS) \cite{internal_revenue_service_annual_2016} and the UK's Charity Commission \cite{hm_government_charity_????}. The focus of these systems is typically on \textit{input} (i.e. the money a charity spends). This is, in part, due to its ease of measurement; however such input transparencies \cite{heald_varieties_2006} have been shown to be ineffective when determining how appropriate a spend is. Previous work in HCI has critiqued such systems for lacking detail and context about the work of organisations, and failing to represent non-monetary elements such as the efforts of those who volunteer for a charity's projects and cause. The recommendations from this prior work are for new digital systems and processes that provide a more comprehensive and value-driven alternative to simple financial accounting \cite{marshall_accountable:_2016}. Additionally, imposed or expected transparency measures are often seen to be in conflict with effective practice with regards to organisational independence, confidentiality, and privacy \cite{cukierman_limits_2009, schauer_mixed_2014}. In this way, the concerns of charities around transparency touch upon personal concerns around privacy that are addressed by McAuley et al.'s  Dataware Manifesto \cite{mcauley_dataware_2011} as charities wish to communicate an accurate view of their work and its value but may have concerns presenting data about activity or spend that can be misinterpreted by others who may not understand its context. Furthermore, on a lived and pragmatic level, `being transparent' can create additional work for organisations due to the effort involved in performing audits, monitoring and reporting that they are legally or conractually obliged to. It also means charities have to expend further effort to communicate their practice and value (as opposed to values) in order to maintain a relationship with their stakeholders \cite{macmillan_relationship_2005}.

% What remains to be seen is how transparencany is produced in an organisation, what accounts are given, what interactions this enables (or doesn't), and ultimately what future systems can do to facilitate this.
The work reported in this paper builds upon previous studies in HCI around the design of systems to facilitate transparency and accountability in charities \cite{marshall_accountable:_2016}, and work that discusses the use of data for interaction by and between individuals and organisations \cite{crabtree_human_2015, elsden_resviz:_2016}. Where previous work investigates the design for interfaces to interact with data, or the ownership and processing of the same, it typically fails to account for the work needed in organisations to compile this data in the first place. As such, our research set out to ask: how is work performed and money spent; how is this accounted for in a charity?; and what are the processes that make this available to others? In asking and examining these questions through ethographic fieldwork, this research seeks to provide insights around the ways in which digital systems can be designed to facilitate the work of `being transparent' as part of everyday practice in charities.

\section{Study Design}
Our fieldwork was conducted over a period of seven months with a small charity in the UK based in the city of [anonymous for peer review]; Pallet (a pseudonym for the purposes of reporting the research). The organisation has three full-time and four part-time staff (Table~\ref{tab:participants}), and an average financial turnover of approximately \textsterling 130k. Pallet are a youth organisation that perform work with young people aged between 8 and 25 years in a catchment area typically seen as experiencing tensions related to economic deprivation and a high immigrant population. The participating organisation was approached due to a recommendation of a collaborator who represents the third-sector across the entire region; Pallet were presented as an organisation who have a significant presence in their community, whose work is value-driven, and are exemplar of small charities with flexible funding. They were also presented as being enthusiastic about becoming involved in research of this subject and scope.

Fieldwork and data collection were primarily ethnographic in nature \cite{crabtree_doing_2012}, formed of participatory-observation activities at Pallet. This involved shadowing, assisting with accounts preparation, and interviewing members of staff, volunteers, and service users in-situ. Near the conclusion of fieldwork three workshops were conducted that acted as an opportunity for members of the organisation to engage in checking and elaborating on the emerging insights of our fieldwork, as well as stimulating further reflection on their practices and their use of data and technologies.

% \begin{table}
%   \centering
%   \begin{tabular}{r c c l}
%     \toprule
%     & \multicolumn{2}{c}{\textbf{Participant Details}}} \\
%     \cmidrule(r){2-4}
%     {\small\textbf{Name}}
%     & {\small \textit{Gender}}
%     & {\small \textit{Pay Scheme}}
%     & {\small \textit{Role(s)}} \\
%     \midrule
%     Mike} & M} & Full-Time Employee} & Manager} \\
%     Andi} &F} & Full-Time Employee} & Senior Youth Worker}\\
%     Dean} & M} & Full-Time Employee} & Youth Worker}\\
%     Chelsea} & F} & Part-Time Employee} & Adminitrator (pre-July)}\\
%     Lynne} & F} & Part-Time Employee} & Admistrator (July - present)}\\
%     Sonia} & M} & Part-Time Employee} & Youth Worker}\\
% 	Ladislav} & M} & Part-Time Employee} & Youth Worker (pre-August} \\
%     \bottomrule
%   \end{tabular}
%   \caption{Details of Participants.}~\label{tab:participants}
% \end{table}

\begin{table}
  \begin{tabular}{llll}
    \textbf{Name} & \textbf{Gender} & \textbf{Pay Scheme} & \textbf{Role(s)} \\
    Mike & M & Full-Time Employee & Manager \\
    Andi & F & Full-Time Employee & Senior Youth Worker\\
    Dean & M & Full-Time Employee & Youth Worker\\
    Chelsea & F & Part-Time Employee & Adminitrator (pre-July)\\
    Lynne & F & Part-Time Employee & Admistrator (July - present)\\
    Sonia & M & Part-Time Employee & Youth Worker\\
    Ladislav & M & Part-Time Employee & Youth Worker (pre-August) \\
  \end{tabular}
  \caption{Participant Details}
  \label{tab:participant}

\end{table}

\subsection{Ethnography}
Our ethnographic work began in early 2016. Initially, the fieldwork came in the form of weekly visits by the lead author to Pallet in order to participate in their daily administrative and planning sessions. At this time, visits were targeted to coincide with the shifts of the part-time administrator so that the lead author could engage with their work as well as that of other staff members.

After three such visits, fieldwork expanded to include participating in the organisation's work as a volunteer youth worker on a weekly basis. This further facilitated the lead author's integration into the charity, and provided opportunities to participate in and observe Pallet performing their work in the community in order to develop a deeper understanding of their practices. Through this participant-observation, the lead author was able to develop a vulgar competence \cite{crabtree_doing_2012} of organisational processes from which to learn from and reflect upon. At this point, vists became more frequent and occurred several times a week with days being spent partly participating in administration and planning, and partly in the performance of a volunteer role in community sessions and projects.

During this time at the organisaiton the author was given a range of duties to perform such as: everyday purchasing of equipment for activities; attending meetings of stakeholders; being involved in strategy meetings with partners; the creation of monitoring materials such as questionnaires; and reconciling financial accounts. The author was also given copies of the yearly accounts spreadsheets to inspect at leisure, with instruction to ask any questions as required. Informal interviews often occured in-situ, either when the researcher desired clarification of an activity as it occured in-the-moment, or when reflection on fieldnotes lead to a question which could only be answered by the settings members. These informal interviews were not audio-recorded, although integrated into the data corpus through fieldnotes and fieldwork diaries.

The seven-month ethnography comprised 49 unique visits and engagement in 27 volunteering activities. Some 70 pages of fieldnotes and fieldwork diaries were generated, and elaborated on with discussions with settings members, as activities occured.


\subsection{Workshops}
In the final stages of our fieldwork, a series of three workshops were conducted with workers at Pallet. During the first workshop, four staff members and one volunteer staff participated. Only three of the staff were available for the second workshop. However all four participated in the final workshop, alongside one trustee of the organisation.

The purpose of these workshops was to provide a setting in which the workers could reflect further on their practices as an organisation and discuss the role that technologies and data play, and might play in the future, in the daily operations of the charity. The workshops were also an opportunity for members to check the preliminary insights from our fieldwork, and to offer further elaboration on specific points. In total, three workshops were run at three-to-four week intervals. The workshops each lasted an average of 105 minutes, with the first being slightly longer than the subsequent two due to the personal schedules of participants. All workshops were audio-recorded (with permission) and written transcripts were produced.

The first workshop was designed to elicit reflection on how the organisation communicated its spending and activities internally as well as with stakeholders. Participants engaged in a collaborative process using common office materials (e.g. corkboards, post-its) to build up a rich picture \cite{monk_methods_1998} of Pallet and its ecosystem. First participants noted individual elements of their processes including: staff; stakeholders; assets and resources; government bodies; funders; and beneficiaries. Participants notably chose to represent these elements on a loose spectrum from \quoteit{inside [Pallet]} to \quoteit{outside [Pallet]}. Once these elements had been added, the researcher prompted discussion of information flows around Pallet between different elements, using examples derived from fieldwork. These were captured on the rich picture using coloured string and map pins to represent different flows and activities. After some time doing this, the researcher moved the discussion to the flows of financial capital. This portion of the workshop followed a similar process, using examples derived from fieldwork to prompt reflection on these flows, again which participants mapped onto the rich picture using coloured string and pins. Finally, the session concluded with a targeted reflection on tension points the participants felt were identified in the workshop.

The second workshop had the purpose of exploring in further depth the relationship between the participants and data about the organisation's finances and work practice. In this workshop, participants were first given nine scenarios written by the lead author, which imagined a variety of potential interactions with data about the organisation's activities. The participants chose to read each scenario aloud in turn, and discuss each one before moving to the next. After each scenario was discussed (approx 60 minutes in total), the participants were asked to design their own vision for supporting accountability and transparency in the organisation. Participants were told that their designs did not need to be technologically feasible, and could resemble `magic buttons' to do what they wanted. Participants, again, opted for a group dynamic where they spent a brief time sketching or writing, and then discussing the ideas in depth between each other and the researcher.

The third workshop was designed to allow participants to build further on the concepts that they'd discussed in the previous workshops. Based on a design brief given by the facilitator, participants used commonplace materials to construct `magic machines' that an imagined future version of Pallet would use to perform tasks related to transparency and accountability. After approximately 45 minutes of building individual machines, workers took turns to explain them through demonstrating their use around the setting (the charity's main office). Here they would physically take the researcher and other participants to areas in which the machine was imagined to operate as they elaborated on its functions and their interactions with it.

\section{Findings}
Our findings are compiled from field notes and diaries collected during the lead researcher's immersion at the organisation, as well as transcripts from the audio recordings made during workshop events. These were used to develop praxeological accounts of work \cite{crabtree_doing_2012} around the organisaiton's activities related to producing accounts of their work, both financial and otherwise.

The findings presented here are grouped based on the activities they relate to: Accounts of Finances; Accounts of Outcomes; and Accounts of Unproductive Labour. \textit{Unproductive Labour} here refers to the effort required by the workers of the charity to make their work productive \cite{marx_contribution_1970}, and we concern ourselves not only with how this is performed but how this is accounted for and communicated to others.

\subsection{Accounting for Spending}
Spending from funded projects occurs on an everyday basis for things that are required to achieve the charity's immediate outcomes and goals, as well as having core costs such as staff salaries and bills which are paid through other processes. In this section we describe how the charity account for everyday spending, staff and core costs, as well as what is involved in producing `the accounts' required by legal processes.

\subsubsection{Everyday Spending}
One of the first things we observed during fieldwork is that everyday spending is accounted for internally by instituting a system of funelling all spend through two senior members of staff. Chelsea, the charity's part-time administrator, described this in the fieldwork:

\textbf{Chelsea} \quoteit{The staff get paid back through expenses, and only Mike and Andi are allowed to make expenses claims which they'll make generally when they notice their bank accounts are getting low}.

Chelsea here indicates that Mike and Andi use their personal bank accounts to make purchases for the organisation in order to ensure that any expenses made can be deemed appropriate. That they only claim for expenses when they \quoteit{notice their bank accounts are geting low} also indicates that there is a practice of storing records of these transactions in order to compile them and reimburse the spender. We also observed that although all spending must be funelled through either Andi or Mike, often other staff members made purchases as part of their everyday work. This occurs in one of two ways, both of which we describe in a vignette below, which describes events occurring over two days of fieldwork:

\textit{I had been called to help with the planning and execution of the `Community Activity Day' at the Play Centre. The day itself was taking place on the first of June, but I was participating in the planning on the day before. After I had arrived, Sonia and I were tasked with producing a grocery list for an outdoor BBQ that was occuring as part of the day. After a quick discussion regarding items for the menu, we needed to come up with quantities. I asked Sonia what the expected turn out was and she said ``Probably at least 70''. We agreed to plan for around 80 attendees and proceeded to leave the centre towards the supermarket, where we were approached on the side of the road by Mike in the minibus. He asks us if we're ``off to buy food'', and when Sonia replies that we are he says ``Here, take this'' and hands his debit card over, before asking Sonia ``Do you know the PIN?''. She nods affirmatively and Mike chuckles before saying to me ``Aye. Half of [district area] know that PIN now'' and driving off. When we are in the supermarket, Sonia and I traverse it looking for the items we need --- Sonia makes a point to purchase the cheapest possible store-brand products, searching for them on the shelves and looking frustrated when none appear available. When I ask if it's possible to get one of the others she tells me ``We can't be seen to be buying brands really''. After we have everything in the basket, we purchase it using Mike's card. Back at the project hub, Mike returns around an hour later during lunch and retrieves his card and the receipt of purchase from Sonia, checking over it briefly before putting it in his wallet.}

\textit{On the day of the `Community Activity Day' I was walking towards the Play Centre in order to volunteer my labour, when Mike pulled up in the minibus heading in the opposite direction (at speed). He stops only to hand me \textsterling 20 and tells me ``We need [toilet roll] for the Play Centre. Go get some from [convenience store] across the road, the big cheap pack at the back of the shop'' before driving off. I turn around, find the store and make the purchase, and then make my way to the Play Centre. Inside, the Play Centre is already full of activity with adults and children moving around. I find Andi and hand her the money, which she takes and asks for a receipt. I find the receipt in my pocket and hand it to her and she stores it together with Mike's change in her back pocket before assigning me my role for the day.}

This vignette first illustrates the ways in which spending is funeled through the senior staff whilst allowing the organisation to distribute the labour of purchasing by devolving responsibility for the physical purchase. Sonia is handed Mike's debit card so that it is \textit{his money} that is spent, and this acts as a buffer between the member of staff and the organisation's finances. This buffer is also present when Mike hands cash to the researcher so that they can participate in spending. Further to this, that Mike checks the receipt, and Sonia's awareness of not wishing to be \quoteit{seen to be buying brands} shows a checking process that means that Sonia may have to justify purchases to Mike if called upon. This also shows that the staff involved in purchasing are aware of the charity's overall budget and tailor their purchases to ensure it's appropriate; also seen when Mike explicitly provides the lead author with instructions to purchase \quoteit{the big cheap pack} of toilet roll.

\subsubsection{Staff Salaries}
Table~\ref{tab:participants} lists the staff at Pallet. There are three full-time staff: Mike, the manager; Andi, the senior youth worker; and Dean, a youth worker. The three part-time staff are: Lynne, the administrator (replacing Chelsea who had the role at the start of fieldwork); Ladislav, a youth worker; and Sonia, a youth worker. When ethnography started, the administrator role was performed by Chelsea.

% Vignette and quotes from Chelsea early on re Mike getting paid part time
During discussion with Chelsea at the start of fieldwork, we discussed with her how staff are salaried at the organisation:

\textbf{Chelsea:} \quoteit{Dean and Andi get paid full time, I get paid part-time for about 16 hours a week. Mike works full-time but he's only paid part-time.}

Chelsea lists several of the staff and their pay-schemes, but noticeably says here that Mike is working full time but only paid for part of his work, indicating that his salary is variable even though his role is central to the organisation. During a subsequent fieldwork session, Mike elaborated on this when asked:

\quoteit{It's what's best for [Pallet]. We have to keep things going. I don't care how much I get paid, and it's money that I have to end up looking for. I put the salaries down for the last few years, and it took a while to put Dean up to 20,000 when he started because the money just wasn't there. What with the Lottery Fund coming in now we can start thinking about putting the salaries back to normal.}

Mike's discussion of the staff accepting lower pay provides insight into the values of the organisation. The staff are dedicated to the organisation's work, and are aware of their impact on its finances; accepting lower pay in order to \quoteit{keep things going}. Where Mike discusses having to look for the money to pay staff, he also touches upon how an increase in pay creates an increase in labour as he is required to expend effort sourcing funds to make up the difference. Further into fieldwork, Mike provides additional insight into this during discussion about staff salaries and standard pay increases amid the adjusted salaries:

% -->>> Mike's discussion of finding money vs the tax and benefit to him <<<--
\textbf{Mike:} \textit{``So we're putting the salaries up starting next month which is a big relief for everyone. I'll be on 30k, but not really because that means more tax so you have to judge it carefully. Because of the tax brackets, past a certain point it makes no sense to give me a pay increase because of how much it'll cost. What is an extra hundred to me per week will be several thousand a year to the charity which I then have to find and justify finding. This way everyone still sees their pay increase, including me, but I'm not too worried about finding the extra cash. It's still the least you'll ever see another project manager get paid round here though. Some of the larger organisations have six or seven heads on about 100k so that's like nearly a million you need before you even get anything done.''}

This again shows Mike's awareness of the staff salaries having an impact on the organisation; he is willing to keep his salary lower than that of comparable positions in the area (\quoteit{round here}) and demonstrates that he would need to justify to others a pay increase that required searching for a disproportionate amount of further funding. Mike also mentions how the staff will be relieved that the salaries are being brought in line with standard pay rises; illustrating that the salary cuts have tangible effects on staff and further defining their position as a value-driven cohort. When Mike discusses the salaries of larger organisations he also reveals his views on what money and people are supposed to do in an organisation; they are supposed to be put towards the work of the organisation and paying head staff large salaries creates extra work and financial pressure \quoteit{before you even get anything done}.

\subsubsection{Compiling Accounts}
% Outline the flow of financial information through the accounting system.
All of the organisation's income and spending must be accounted for formally through the compilation of `the accounts'; records of financial transactions that must be reconciled, audited and presented to governing bodies. This compilation of accounts was one of the activities we involved ourselves in during fieldwork, generally performed alongside the administrator (Chelsea, and later Lynne). When being initially instructed in the task by Mike, we were given insight into the role of financial accounts in the organisation and what is involved in the task:

\textbf{Mike:} \quoteit{We have this budgeting tool. It's an Excel spreadsheet really, do you know Excel? Anyway this Australian lad who used to work for us set it up, we can add funders and add spending and stuff and we can use it to see how much we have left in each budget. At the end of each financial year this gets sent to the accountant so they can sign it off for us.}

This example encapsulates how the organisation views using the spreadsheet when doing budgeting; Mike refers to it explicitly as a tool, with which he can present an account of the organisation's budget to himself, and can be used to generate another account to others (one which is legally stipulated). We did, however, witness that there are often tensions arising from Pallet's use of their spreadsheet. We describe this in a vignette below:

\textit{It was the start of the day and we were having a scrum, after everyone says their job for the day Mike asks to speak to me about doing something with the Patchwork accounts. ``I'm not happy with the accountants at the moment, they're being problematic''. I ask why and he responds that ``They just want us to use Sage [business accounting software], do you know Sage?'' I respond that I don't, and Mike continues ``The accountants don't like that we don't use Sage, and I think that's because they can just import it and have it do their job for them.''}

Here Mike shows us that there is an explicit point of contention between Pallet and their accountants when the accounts are audited. Pallet have developed an in-house tool that allows them to present an account of spend to themselves and others. Yet because the file cannot be read by the accountant's software then this makes it incomprehensible to them. Mike then elaborates and tells us how the accountants are trying to influence Pallet's practice to suit their own, and postulates that it is to reduce the labour of the accountant. That Mike starts by telling us that he's \quoteit{not happy with the accountants} shows that he feels that the organisation have fulfilled their obligation to produce an account of their spending. Further insight to this was provided during later fieldwork, when the researcher attended a trustee's meeting, and it came to discuss the accounts:

\textbf{Mike:} \quoteit{We're thinking of scrapping [accountants]. They've upped the price to \textsterling 1300, which nobody around the doors is happy with by the way, and they're trying to force us to use Sage so we do their job for them. We've spoken to a woman we found on [a listing] who says she'll do it for \textsterling 20 an hour and she's happy to do them in whatever format we want. She's been in and looked already and she's told us that we've already done the job, and all she'll need to do is double-check a few things and sign it off. We have to make sure she's got the right, y'know, qualifications, to do that but aye it looks much better.}

This extract shows us that Pallet are willing to seek other accountants to perform the task who are more suited as partners as they might cost the charity less money for their services but more importantly are flexible and accepting of Pallet's accounts regardless of format. Additionally, when Mike discusses both the price increase in reference to other organisations (\quoteit{nobody around the doors is happy with by the way}) and checking the new accountant's qualifications, he also demonstrates how they view their relationship; the accountant must become accountable to them and the other charities that the accountant works with). This holds true for the current accountant in terms of their service cost, and the potential new accountant having the correct qualifications for the job.

\subsection{Accounting for Outcomes}
As well as having to account for their financial spending, Pallet also produce records of their outcomes which they demonstrate to others. We saw many examples of this during fieldwork, and present an account of one here:

\textit{During a community play session in the park, I saw that Andi was often taking photographs using her phone. Whenever she wasn't actively engaged with young people, she would navigate to the others in the group and attempt to take a photograph of them; often to a mixed response from the young people themselves. Several of the group were very willing to be photographed, usually the young Slovak boys who took up exaggerated masculine poses such as scowling or muscle flexion. Others were more reserved, and stipulated that they only be photographed alongside the others, so that they were not the sole subject of the picture. Whenever possible, Andi would call to another youth worker such as Dean or Ladislav and ask them to get into the photograph as well. I am photographed with young people several times. The next morning, when I wake up, I have been tagged in photographs by Pallet's Facebook account and see that most of the workers have also been tagged as well as the majority of young people in the photographs.}

Andi's behaviour shows her producing a qualitative record of the event and activity that occured. She can be seen collecting photographic evidence of their attendance in-situ, as well as using the setting to elaborate on the context of their work. The practice of uploading these to a social media profile produces an account of their activity for others to see, and the tagging of people in the photographs on the platform encourages those tagged to look at them and potentially allows others (such as parents) to glimpse the activity as well. As well as on social media, Pallet print out a selection of photographs in a poster format, which are displayed around their main community hub. The workers reflected on this practice in the first workshop:

\textbf{Andi:} \quoteit{... part of it's capturing that moment in time because it's gonna be gone. Y'know, and it would be very easy for them to forget [...] So you're capturing it for them, you're capturing it for their parents to see what they've achieved, or for the Duke of Edinburgh so they can prove whatever it is they've done. You're putting on the wall as a celebration, you're putting it in the annual report for funders to see and also for young'uns to see [...] Like loads of kids will be like `will this be going on the wall?'.}

\textbf{Mike:} \quoteit{I mean we dunno we just take lots of pictures because it becomes a resource for us as well. So the ones on the wall are of the D of Es because they're positive images. Sitting down two people and talking one to one and that --- it's not very entertaining.}

We see here how the organisation use a resource bank of records built up by photographs for different types of accounts, to different people. Andi relates how the photographs she takes can explicitly be used as evidence for the young people's involvement in the \textit{Duke of Edinburgh} award, whereas Mike conceptualises them as \quoteit{positive images} and a resource for the organisation's future needs. Andi also explicates how the photographs are shown to parents in order to provide an account of their child's activity (achievement) with Pallet. As well as this, we see how the photographs are repurposed to provide an account of value in the annual report, and to provide a personal record for the young people when it's placed on the wall in \quoteit{celebration}. The ability for these records to form a resource from which different accounts can be derived also sits in contrast to other forms of work that Pallet perform that, as Mike indicates here, are more difficult to account for (\quoteit{Sitting down two people and talking one to one and that --- it's not very entertaining}). We observed this first-hand during fieldwork when Mike expressed frustration at the records that Pallet are required to keep of their meetings with service-users, and how it is difficult to present these to others:

\textit{I followed Mike through to the back office, and up to a large metal filing cabinet that was unlabelled. He opened it and took out a folder to show me an example, flicking through saying ``Here. This is a monitoring form we have to fill out every time we have a chat with someone. You say who it was, what you chatted about and what the outcomes were. Standard ticky-box stuff. We're meant to keep this, and we do by the way, but nobody ever asks to see it. I've got files here from ten year ago which haven't seen the light of day. And people complain at us that we're not doing our job and ticking boxes and we are, but nobody ever comes in. Nobody ever asks.''}

Mike's frustration indicates that although he is fulfilling legal and stipulated obligations designed to make them accountable for their work, they are not given an opportunity to actually do so. When Mike describes above that photographs of these chats would be \quoteit{not very entertaining} we also see that whilst Pallet could theoretically generate records of these to show to others, the effort required to do so would not result in a substantial gain for the charity when trying to demonstrate their value.

\subsubsection{Accountability inside and outside of work}
We found that the workers at Pallet saw themselves as being accountable both in their roles as youth workers, but also as members of the community. This is characterised by Dean's conception of accountability during discussion in the second workshop:

\textbf{Dean:} \quoteit{Then there's the visibility in and out of work. It's not a one-way thing, I'm not Dean the youth worker during the day and I'm not Darts-Dean at night I'm both and I've got to be very aware that young people and the families that I work with, [...], I live in the same area as them and they are watching me constantly. In [Pallet] and out. I've got to be visible. It's... an awareness of your role within the community. And I think another one for me, being accountable is remaining humble and just thinking that I'm very much where I've come from and I'm very like the young people I work with and they know my family.}

With this, Dean shows us how he sees his role in the community by living and working in the same area. Dean provides a view that accountability for his actions as a youth worker is lived in each moment. He is constantly watched by those around him, even when outside of work during his recreation activities and can therefore be seen as a whole, rather than only through a lens of his output at Pallet.

\subsection{Accounting for Unproductive Labour}
Unproductive labour is the labour required to perform the work to achieve the outcomes for the charity. In other words it is the work needed to make work productive \cite{marx_contribution_1970}. In this context it refers to the additional effort expended by workers at the charity in addition to what the task demands in-the-moment. An example would be the planning required to execute a community session ahead of the labour required by session as it occurrs. We found that accounting for unproductive labour in the organisation occurs only as the labour itself occurs, during meetings, or discussions about activities and planning --- and that it is rare for those outside of the organisation and immediate community to be made aware of this labour. This is often complicated by the organisation's open-door policy, which requires that they respond directly to community members coming through the door for their services or informal chats. This particularly came to the fore in one discussion during fieldwork:

\textit{The workers were sitting down at the central table, discussing another youth project that operated elsewhere in the city. Pallet have recently acquired a Play Centre and are in the process of finding ways to use it most effectively, and so have been visiting other charities to learn from them. It's mentioned that the other project in question produce elaborate and planned evenings of activities for all attendees at their centre whereas Pallet wouldn't consider that. Dean sits up straight and exclaims ``They've got the time they don't start until half four! As soon as that shutter goes up we have work to do!'' As he says this he gestures with his pen at the large window towards the front of the room, where Pallet's hub faces the street. The rest of the group at the meeting nod silently in agreement}

Dean is primarily discussing how Pallet's work cannot be judged against that of another organisation with different working patterns. He also makes reference to the open door policy and its effect on their working day regarding planning. He also makes clear that these informal meetings are conceived of as `work'; there is effort expended when conversing that prevents them from performing other tasks. These conversations must be engaged in because they also form an important part of how Pallet organise their work. This was discussed during the first workshop:

\textbf{Mike:} \quoteit{Aye it's the Youth Worker, the Young People as well as it starts and it's an outcome from the process of that conversation really. But it can be other things}

\textbf{Andi:} \quoteit{So aye, [anon] is a good example. [...] I know he was doing football, I knew he was doing work experience so he'd have the time and you just think well it would be really good for him to do it for his future. Y'know, so having a conversation with him to say look are you interested in this [...].}

This shows how the process of engaging in conversations that arise from the open door policies can translate to outcomes, in this case a beneficiary getting a work experience placement based around a hobby. This qualifies Dean's earlier utterance that the organisation has \quoteit{work to do} as soon as they start: these conversations are labour that must occur for Pallet to achieve its goals effectively, but it's difficult to provide an account of this work to others. During fieldwork, Mike related another anecdote that illustrates how outsiders are often surprised at the \textit{labour} required to perform everyday tasks and achieve outcomes:

\quoteit{It's like when this guy from [a funder] came in to check. Most funders don't and they don't understand us. He came in and he loved it. He said that he was amazed we could keep the place running, we had so much going on around here that we deal with on a daily basis.}

From this we also see that Mike understands the difficulty of accounting for this labour to others --- most funders do not visit and thus do not understand how the project functions. The funder is also amazed at the everyday work that occurs and was surprised at the effort being expended; which shows that this labour is not represented to them through other means and becomes accounted for only when there are others there to produce their own account.

\subsubsection{Inferring Unproductive Labour}
We mentioned that unproductive labour is generally only accounted for by those performing it as it occurred. However we saw during fieldwork that unproductive labour can often be inferred by other records that are either kept by workers or produced as a by-product of their activity:

\textit{I was participating in a planning session for the evening's activities which was initiated when Dean and Andi each took out large workbooks and opened them. Andi asks ``Where's Mike?'', to which Dean responds that he is ``down the allotment''. Andi looks puzzled at this and Dean elaborates, ``He's seeing how Liam's getting on'' and turns the notebook to show Andi. On it there is a task list which shows `allotment'. Andi looks at this, and nods.}

This illustrates how records kept by workers can be used to infer the activity and thus the work of others in the organisation. The list offers proof of Mike's whereabouts that Andi accepts as canonical, yet only the word `allotment' is written. This tells us that, generally, there is labour that is performed at the allotment and that Mike can be found performing it. In addition to this we see that, in addition to being necessary for financial accounting, receipts can also be used to infer the labour of others:

\textit{Mike was having lunch and moving items from the tabletop out of his way, so that he could place his laptop there and write a report. When moving a pile of paper he turns to inspect it and finds a receipt, saying aloud ``What's this? Ohh. It's the pancake stuff for tonight; Sonia's been shopping.''}

From the receipt Mike can recognise that the items are a list of ingredients to make pancakes, an activity commonly run by the charity. He also infers that there has been an expense of labour in acquiring these materials when he says \quoteit{Sonia's been shopping}, and can attribute this to Sonia by other mechanisms that he is privy to.


\subsubsection{Hidden Unproductive Labour}
As discussed, unproductive labour is necessary for the subsequent performance of activities that translate to outcomes. Whilst most of this labour can be inferred by the activity, or made visible to others through things such as task-lists, some of it cannot as it takes a form that doesn't lend itself to recording. We illustrate this with a vignette of activity in the organisation leading up to a scheduled evening activity:

\textit{I was due to participate in an evening session with a group referred to by the workers as the `Slovak Lasses' group. This group was comprised of young Slovak women aged between 15 and 24. The `sessions' all run from 1600 approx until about 1830, and the plan is to run a BBQ event for the attendees. From about 1545, two of the `lasses' had turned up alongside one of the part-time workers and positioned themselves on the end of the row of computers on one wall. They are checking Facebook actively. On another computer, Dean is on Facebook using the charity's account and has several chat windows open. When prompted, Dean responded that they were ``chasing up'' the rest of the lasses to make sure that they were coming. As she passed through to go to a meeting elsewhere Andi convinces the two Slovak lasses to accept her taking a photograph of them on her way. Dean signs off the computer at 1630 and returns to the central table. At 1655, there is no sign of the lasses and Dean is visibly concerned, pacing back and forward. He mutters to the room that ``we should sack this group''. Sonia agrees looks at me and Dean in turn and then says ``this is ridiculous. We have two young people and four staff''. I am dismissed by Dean who turns and says directly to me ``You can go if you want. It's a bit weird if we outnumber the girls and we have loads of staff in''}

This example illustrates two things which are problematic for the accounting of unproductive labour at the organisation. First, that the unproductive labour required for the execution of the outcome is occasionally hidden and cannot always be inferred from the activity. As we've previously discussed, the execution of an event often entails unproductive labour to ensure that it can occur in the first place, and that this is accounted for by the workers through processes of internal task lists or status updates and can be inferred through records of spending generated such as receipts. What is not immediately transparent is the existence of Dean's task of having to `chase up' the beneficiaries of the event, expending additional effort to make the event possible on top of the labour that has already occurred; and this labour cannot be discovered or revealed by any means that the organisation use to account for time or effort. Second, the outcomes often don't reflect the scale of the labour input by the workers. Dean spends several hours on the computer engaged in conversation to the lasses, and labour has already been expended organising the event with the attendees prior to the night --- whereas only two beneficiaries can be accounted for during the activity. Dean's utterance of \quoteit{we should sack this group} indicates as well that this lack of commitment from the lasses is not an uncommon occurrence and thus that the workers often input labour to the group that is not reflected in how the success of the outcome is measured is problematic. Additionally, Sonia indicates that several staff have committed labour to the event (\quoteit{We have two young people and four staff}) which further skews the mapping of input to outcome somewhat when it is accounted for.

\section{Discussion of Findings}
% JV COMMENTS === Of course though it is the discussion that really needs work still. It seems to me that what you really haven’t done yet is translate what you have learned about this one organization to (1) the bigger picture of accountability for these types of organizations more generally (and the problems with that) and (2) what these means to “build some f’ing systems”. I have suggest in one of the comments a range of things that you can draw out more; I would focus on some of the interesting practices you have observed (e.g., PIN sharing; distributed expenditure; capturing unrecognized work) – explain why they are interesting (e.g., someone giving their personal bank account details out and this being seen to be normal); why they are problematic system wise (e.g. PINS exist to hold that one person accountable for their expenditure and get banks out of shit when things go wrong); how this may relate to prior work (e.g., work on PIN and password sharing more generally and the problems with collaborative finance in a one-person-per-account world or where joint shared accounts are problematic) and then tell us something that we can take on board around this (no idea).

% JV COMMENTS === So, having read the findings, and not read the below, I am expecting a range of overlaps with prior HCI work and building on what you’ve done here – brain dump. That the stuff around the informal exchange of information, passwords, PINs etc. touches on the work I’ve done in HCI before and the folks on password sharing around how “formal” protocols of security frequently get negotiated and worked around when there is a need for flexibility or multiple access to single accounts. That there is something here around values-oriented accounting in certain sectors (as opposed to value oriented accounting). That expertise around “getting a good deal” and “thrift” could be recognised more as an outcome measure. That there are connection with work on social media as a data trails and ‘life logging’ that can be used to build some ideas around qualitative accounting. Stuff around the tensions inherent in not recognizing invisible work that goes into planning activities and events; but recognizing this might actually foreground project failures even more.


%% What we found
Our findings demonstrate how the charities experience their accountability in terms of their values and labour as individual workers and as an organisation, and how their interactions with technology can shape their interactions with those they are accountable to. We find that the organisation and its workers engage in everyday interactions and processes that they have designed to make themselves accountable by requiring them to justify their actions, and make them answerable \cite{fox_uncertain_2007} to each other and their stakeholders in the immediate community. Accountability, and financial transparency, is lived in-the-moment and affected by their relationships as individuals to each other and the communities in which they reside as they respond to their needs
\cite{koppell_pathologies_2005}.

The ways in which this lived accountability manifests is also subject to the data technologies that can be used to capture practice. We see how accounting for outcomes in charities resembles the practices of storytelling seen in previous HCI studies \cite{erete_storytelling_2016}. Meanwhile the only way to capture and account for effective financial practice is to become subject to the transparency measures known to have little impact on work value, namely accounting for spend \cite{heald_varieties_2006}. In this way, both values and effort are hidden to the outside, but can be accounted for to those closest to charity and who benefit directly from its work. As such we see that organisations express their notions of accountability and transparency through the socio-technical systems which are constrained by the characteristics of systems that they have to hand. This in turn implies that systems deployed in this space have signficant effects on how organisations see themselves as being accountable.

Previous work in HCI also proposes new ways for systems to account for the work of organisations in this space \cite{marshall_accountable:_2016} and speaks to the danger of focusing on the flows of financial capital over the organisation's outcomes. We build upon this discussion now by presenting the ways in which the working practices of those in charities further affect the design of systems that facilitate transparency and accountability with regards to work performed by charities and the value that work has. Whilst the value of ethnographic work should certainly not be limited to providing design implications for systems \cite{dourish_implications_2006}; our work's purpose to explore charity working practices and illuminate how these \textit{do affect system design}. As such we turn now to explicating the characteristics of systems that seek to operate in this space.


\subsection{Hegemonise}
Workers at charities can often see themselves as being accountable to members of their immediate community for their practice and their spending, as opposed to others who are removed from this work but can stipulate responsibility to document it \cite{koppell_pathologies_2005}. Systems put in place by these actors must be engaged with, although we see that working practices can be adjusted to navigate around formalised measures designed to provide accountability. We note in particular that the behaviours workers engage in to become accountable for financial spending could be considered subversive in terms of system design; most noticeably the sharing of access tokens (PINs and debit cards) to personal bank accounts. This touches on prior work in HCI \cite{vines_eighty_2011} which demonstrates how formalised security protocols in systems such as banking are negotiated in settings where there is need for a flexible approach in order to distribute to labour of purchasing securely using social accountability, rather than that imposed by digital systems.

We suggest then that digital systems in this space need to acknowledge and interact with imposed systems and formats, and hegemonise them in addition to this. Hegemonising in this sense refers to the fact that organisations have practices that they have designed to make them accountable to those around them, regardless of the policies and systems imposed from above. Therefore new digital systems in this space should afford organisations the ability to focus on their everyday accountability whilst being able to produce required documentation as a by-product. This allows them to become accountable to those whose opinion they care for \textit{in terms that matter to them}, whilst remaining accountable `on-paper' and acknowledging the difficulty of affecting policy in this space.

We see this above with subversion of digital accountability in spending, and this must hold true with systems deployed to provide more qualitative accounts. Qualitative accounts can be envisioned as allowing stakeholders to interact with the value, labour and outcomes of the organisation, but mapping this in a system can be problematic. Digital technologies could facilitate data collection and interaction with labour by creating data trails \cite{weiss_digital_2009} or engaging in practices akin to `life logging', but a system that maps this against outcomes may risk devaluing the effort of workers in this space as we see that the mapping of outcomes and labour is not straightforward. In financial matters, it is difficult to represent the thrift of workers when involved in purchasing, or willingness to sacrifice salary for the organisation. In this we see reflected a discussion that measurement of outcome reduces the ability to perceive the work of an organisation \cite{lowe_new_2013} that presents a point of tension for the production of more comprehensive account. To address this, digital systems deployed in organisations must hegemonise other processes and formats to allow workers to concentrate on their everyday accountable practices.

\subsection{Federate}
% qccounts are federated
If we are to produce technologies that allow charities to hegemonise the processes governing their accountability to others, then the systems we deploy must operate in a manner that allows them to function both independently for single tasks as well as considering the wider ecosystem in which they exist. Producing accounts in charities is naturally a federated process whereby multiple systems interact to produce, process, and present accounts and accounting data to a variety of actors. Producing accounts, and facilitating accountability, with digital systems in charities should also be federated in nature.

% Datawareeee manifesto
Where accounting for outcomes and finances occurs in charities, it consists of producing, processing, and then presenting various forms of data. These procedures mirror processes and models discussed by McAuley et al \cite{mcauley_dataware_2011}, which are notably federated. A federated system could facilitate the capture of heterogenous types of data that are suited for the desires of the organisation, and allow other tools to engage in the processing and presentation of this work. In this way it becomes a kind of \textit{federated toolkit} that can be used to provide accountability through different views of the organisation based on the format and requirement of the consumer.

To illustrate how we envision the form that these federated toolkits may take, we present some characteristics of them here. We imagine them to be lightweight, heterogeneous, and able to function independently as well as in a federated manner. One can envision a mobile application that captures spending for the user associated with time and location, can share this with a web-based piece of accounting software which may also receive and associate data from other sources such as images or video produced using a variety of other applications. Other actors in this federation could process the same data in other ways, tailored by and for those who desire to use them.

These federated toolkits can provide scability and flexibility for organisations to document their practice, outcomes, and values for others. Imposed transparency measures dictate the format in which work and spending is encapsulated, whereas federated toolkits can allow these to be subverted, navigated and even hegemonised by allowing organisations new ways to capture their work and its value. It also can be envisioned that as the work of the organisation evolves, and engages with new stakeholders, this could facilitate an active form of transparency \cite{oliver_what_2004} whereby those interested in an organisation's work can commission accounts that are meaningful to them based off of the data collected, and request new information that has not been captured to date, making them organisation reflexive and responsive to them.

\section{Concluding Remarks}
In this paper we set out to explore how the everyday working practices of charities would affect the design of systems that are deployed in order to make them accountable for their actions and spending. We did this by explicating the how the obligations for a charity to be transparent and accountable are lived and experienced in everyday practice, and how they account for this practice in terms of finances, outcomes and labour. We then discuss how these practices are shaped by the socio-technical systems and technologies that the organisations have access to, and present implications for the design of future systems that facilitate transparency and accountability in charities with regards to their form and operation within everyday practice. We draw out how imposed formats and procedures mean that systems that facilitate new forms of accountability must hegemonise this formats so that charities can subvert these and concentrate on communicating their work and values in terms that matter to them, and how that can be achieved by producing federated toolkits.

Future work for HCI should seek further engaagement with charities to collaborate, develop, deploy and study these systems and how they are appropriated by workers to achieve the goal of operating accountably. The invisibility of labour, and tensions around how this could be presented in relation to outcomes, finances, and values should be considered from the perspectives of both workers and stakeholders in charities. In this way, HCI can address global challenges facing charities and impact the experiences and lives of those who interact with them.
