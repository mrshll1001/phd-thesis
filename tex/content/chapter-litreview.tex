\chapter{Background and Literature}
\label{sec:related}

This chapter discusses the background and literature of the various fields that inform this research. Given the interdisciplinary history of HCI research, and the socio-economic domain of non-profit enterprise, it should come of no surprise that this results in a rich and diverse nexus of perspectives which needs to be accounted for.

% ===================================================================================================================================
\section{What is an NPO?}
% ===================================================================================================================================
% How is this diffferent from social enterprise?
% How is this different from a charity?
% How is this different from "third sector"?

% Why are they important?

%%%% Intro -- Why are they hard to define?

This section explores the definition of Non-Profit Organisations, their roles and activities, and the values of those roles. It also considers some of the various forms an NPO can take, and

The definition of an NPO has historically been problematic, due to the sheer diversity of the both the organisations themselves, and the legal frameworks in which they operate \cite{salamon_search_1992}. NPOs are typically seen to operate in what is known as the \textit{third sector} of the economy, emphasising their separation from public-owned and private business enterprises; however the term is often used interchangeably with \textit{Voluntary Sector}, \textit{Independent Sector}, \textit{Charitable Sector}, and many others \cite{evers_third_2004, salamon_search_1992}. Salamon and Anheier claim that the deluge of definitions pose a problem to the sector, as each term emphasises a particular characteristic of the sector at the cost of downplaying others, which is often misleading. An example of this would be how the term \textit{Voluntary Sector} emphasises the contributions of volunteers in the operation of NPOs, at the expense of organisations or activities that are performed by paid employees of NPOs; Frumkin prefers the term \textit{Non-Profit and Voluntary Sector} for this reason, whilst Lohmann prefers the term \textit{Social Economy} \cite{frumkin_being_2009, salamon_search_1992, lohmann_charity_2007}.
%
%%%%% Par 01 --- So do we have a definition?
The diversity of the \textit{third sector} means that a general definition of NPOs is difficult to generate, however Salamon and Anheier go some way to providing one based off of the structural or operational characteristics of NPOs; which therefore allows their definition to cater for the diversity of legal structures, funding, and functions of NPOs \cite{salamon_search_1992, frumkin_being_2009}. Salamon and Anheier identify five base characteristics common to NPOs in their definition: \textit{Formal}, having been institutionalised to some extent; \textit{Private}, meaning that they are institutionally separated from government; \textit{Non-profit-distributing}, where any profits generated from activities are reinvested directly into the \quoteit{basic mission of the agency} instead of distributed to owners or directors; \textit{Self-governing}, with their own internal protocols or procedures as opposed to being controlled by external entities; and \textit{Voluntary}, where the organisation's activities or management involves a meaningful degree of voluntary participation \cite{salamon_search_1992}. This definition is not without issue: it notably excludes various quasi-commercial entities in the UK such as building societies and cooperatives, as well as important historical entities operating in the 19th century; and its usefulness for identifying factors that affect NPO development (one of its original purposes) is questionable \cite{morris_defining_2000, kendall_voluntary_1996, kendall_voluntary_1996-1, steinberg_comment_1998}. Salamon and Anheier's framework does, however, provide an adequate vehicle through which the characteristics of NPOs can be brought to light. Frumkin similarly gives three characteristics of NPOS which align with Salamon and Anheier's framework: NPOs do not coerce participation; their profits are not given to stakeholders; and they lack clear lines of ownership and accountability \cite{frumkin_being_2009}.
%
%
%%%%% Par 02 --- OK so we know what they are, and that they're diverse. So what do they do? Can they be classified?
As noted, the activities of NPOs are incredibly diverse. Salamon states that NPOs \quoteit{deliver human services, promote grass-roots economic development, prevent environmental degradation, protect civil rights and pursue a thousand other objectives formerly unattended or left to the state}, an insight which admittedly barely begins to describe the wide scope of NPO activity, and speaks to a larger issue surrounding misleading classifications and discussions of the organisations themselves \cite{salamon_rise_1994, salamon_search_1992-1}. In further discussion of NPO classification, Salamon and Anheier state \quoteit{No single classification system is perfect for all possible purposes}, but attempt to explain NPO activities through developing a system known as the \textit{International Classification of Nonprofit Organisations (ICNPO)}; which classifies NPOs into 12 groups based on economic activity, with an additional 24 sub-groups \cite{salamon_search_1992-1}. The resulting classifications themselves are very high-level, with groups such as \textit{Health} and \textit{Social Services}, but the sub-groups begin to hint at the type of activities NPOs engage e.g. \textit{Nursing Homes}, and \textit{Income support and maintenance} \cite{salamon_search_1992-1}.
%
% What value do they have? Why do they do it?
Hansmann writes on the role of NPOs that they often emerge from a \quoteit{contract failure} of the market to police the producers of services, and very rarely exist in industrial sectors \cite{hansmann_role_1980}. According to Hansmann, economic theory dictates that the ability of consumers to: accurately compare providers; reach agreement as to the price and quality of service to be exchanged; and assess the compliance of the organisation to their part of the deal and obtain redress if the organisation has not complied \cite{hansmann_role_1980}. The services offered by NPOs often involve a separation between the purchaser of a service and the recipient (e.g. the purchase and transport of food stuffs overseas), and the inability of NPOs to distribute profits to shareholders removes the incentive (and power) of organisations to reduce direct spend on the service provided -- \quoteit{The reason is simply that contributors [to a for-profit business] would have little or no assurance that their payments to a for-profit station were actually needed to pay for the service they received} \cite{hansmann_role_1980}.
%
Mendel and Brudney discuss the relationship between NPOs and \textit{Public Value}, claiming that its creation is \quoteit{philanthropy at its best} \cite{mendel_doing_2014}. King elaborates on the role of NPOs as

% How do they operate?

%%::::::::::::::::::::::::::::::::::::::::::::::::::::::::::::::::::::::::::::::::::::::::::::::::::::::::
\subsection{What is a Charity?}
%%::::::::::::::::::::::::::::::::::::::::::::::::::::::::::::::::::::::::::::::::::::::::::::::::::::::::
%
%%::::::::::::::::::::::::::::::::::::::::::::::::::::::::::::::::::::::::::::::::::::::::::::::::::::::::
\subsection{What is a Social Enterprise?}
%%::::::::::::::::::::::::::::::::::::::::::::::::::::::::::::::::::::::::::::::::::::::::::::::::::::::::
Traditionally, NPOs have embraced values such as philanthropy and voluntarism in order to advocate for, and to provide the means for, services to their client-base \cite{alexander_adoption_1998}. \bollocks{SOMETHING ABOUT STRATEGY AND ENVIRONMENTAL VARIABLES HERE.} Social Enterprise and Social Entrepreneurship have risen as a response to these challenges faced by NPOS today \cite{dart_legitimacy_2004}. Dart broadly describes Social Enterprise as \quoteit{significantly influenced by business thinking and by a primary focus on results and outcomes for client groups and communities} \cite{dart_legitimacy_2004}, Dees states that Social Enterprise \quoteit{combines the passion of a social mission with an image of business-like discipline} \cite{dees_meaning_1998}. In practice, this often includes activities and practices that include revenue-source diversification, fee-for-service programs, and partnerships with the private sector \cite{dart_legitimacy_2004}.
%
Social Enterprise, whilst often regarded as a miracle solution \cite{dart_legitimacy_2004, harding_social_2004, dees_meaning_1998}, is not without criticism. Eikenberry writes that NPOs, focusing on bottom-line expenditure and adopting a social enterprise model actually poses a threat to civil society \cite{eikenberry_marketization_2004}. She argues that the on the bottom line and overhead expenditures forcing NPOs to embrace market values and entrepreneurial attitudes, and changing the way that they operate has been detrimental. Dart claims that Social Enterprise differs traditional NPOs as they generally \quoteit{blur boundaries} between nonprofit and for-profit activities, and even enact \quoteit{hybrid} activities \cite{dart_legitimacy_2004}. This could include activities such as engaging with marketing contracts as opposed to accepting donations, as well as behaviour such as cutting services that are not deemed to be cost-effective \cite{eikenberry_marketization_2004}. Implications of industry funded NPOs notwithstanding \cite{jacobson_lifting_2005}, the shift of effort from effective service delivery to financial strategy impacts negatively on the social capital that is generated. This is through less emphasis on building relationships with stakeholders (previously an essential survival strategy) as service users become framed as consumers, and through market pressures diverting resources towards skills such as project management and away from activities that build social capital \cite{eikenberry_marketization_2004}. Doherty et al. echo this in their description of Social Enterprises, distinguishing them from traditional NPOs by stating that the latter are \quoteit{more likely to remain dependent on gifts and grants rather than developing true paying customers} \cite{doherty_diverse_2006}.


%% ===================================================================================================================================
\section{Why is Transparency important to NPOs?}
%% ===================================================================================================================================
%% What do we mean by transparency?
%% What are an NPO's responsibilities
%% What are their reporting requirements? (To funders, to the public?)
%% What does it mean to be transparent as an organisation?
%% What are the measures taken to ensure NPOs are transparent?
%% How does transparency relate to an organisation's accountability?

%
\begin{quote}
\quoteit{Nowhere, therefore, does transparency get more problematic than in financial reporting for public companies. Going back as far as the 1600s, financial reporting scandals have been the most visible and far-reaching examples of hidden information and outright fraud} - Richard Oliver, \textit{What is Transparency? (2004)}
\end{quote}


Transparency and Accountability are seen increasingly desirable in governments and organisations \cite{hood_accountability_2010, oliver_what_2004, heald_fiscal_2003}. Oliver states that Transparency has \quoteit{moved over the last several hundred years from an intellectual ideal to center stage in a drama being played out across the globe in many forms and functions} \cite{oliver_what_2004}.  Corr\^ea et al. say Transparency and Open Government is \quoteit{synonymous with efficient and collaborative government} \cite{correa_really_2014}, and Steele goes as far to say \quoteit{Transparency is the new `app' that launches civilization 2.0} \cite{steele_open-source_2012}.
%
Non-Profit Organisations (NPOs) in the UK are held to stringent Transparency standards by an organisation known as the Charity Commission, which is responsible for registering and regulating charities in England and Wales \quoteit{to ensure that the public can support charities with confidence} \cite{hm_government_charity_????}. The development of trust is foundational in the relationship between an organisation and those invested in its activities or performance, known as stakeholders, which is compounded by the notion that a stakeholder in an NPO might not be in direct receipt of its services \cite{macmillan_relationship_2005, krashinsky_stakeholder_1997}. Beyond this, accountability is seen as a way of building legitimacy as an organisation \cite{anheier_accountability_2009}. Watchdog organisations such as the Charity Commission and others therefore play an important role in developing stakeholder relationships with NPOs through Transparency measures, making them accountable to those invested in them. Oliver writes that NPO expenditure is often the \quoteit{most emotional}, and a person's decision to invest in a charity will be down to how comfortable and confident they are in its operation \cite{oliver_what_2004}.


Hood writes that \quoteit{Transparency is more often preached than practised [and] more often invoked than defined} \cite{hood_transparency_2006-1}. This section considers various definitions of Transparency in relation to the UK Charity Commission, NPOs, and the measures that are taken to make them accountable to stakeholders. It also inspects Transparency's synonymity with Accountability.

%% ::::::::::::::::::::::::::::::::::::::::::::::::::::::::
\subsection{What is Transparency?}
%% ::::::::::::::::::::::::::::::::::::::::::::::::::::::::
Historically, Transparency was inherent in the actions and interactions of everyday society since, in traditional societies, the density of social networks made one's actions highly visible \cite{meijer_understanding_2009}. Meijer contrasts this with modern societies where \quoteit{people do not know each other -- many people in cities do not even know their neighbors} and argues that societies which operate at a larger scale suffer a decline in social control, which calls for new forms of Transparency that match the scale of the society \cite{meijer_understanding_2009}.


Transparency has been a watch-word for governance since the late 20th century, yet its roots stretch back much further. In a discussion of Transparency's historical context, Hood identifies three `strains' of pre-20th-century thought that are at least partial predecessors to Transparency's modern doctrine: rule-governed administration; candid and open social communication; and ways of making organisation and society `knowable'  \cite{hood_transparency_2006-1}.

Rule-governed administration is the idea that government should operate in accordance to fixed and predictable rules and, and Hood calls it the \quoteit{one of the oldest ideas in political thought} \cite{hood_transparency_2006-1}. Early proponents of candid and open discussion liken transparency to one's natural state, and it saw an implementation in the `town meeting' method of governance where members of the town would deliberate in the presence of one another -- making all deals transparent and ensuring all parties were mutually accountable \cite{hood_transparency_2006-1}. The third of Hood's `strains' of proto-Transparency doctrines, is the notion of making the social world `knowable'; largely through methods or techniques that act as counterparts to studying natural or physical phenomena. Hood describes an 18th-century \quoteit{police science} which exposed the public to view through the introduction of street lighting or open spaces, as well as the publication of information (all of which designed to help prevent crime) \cite{hood_transparency_2006-1}.

\bollocks{more history stuff?}

Heald describes modern Transparency as directional; with the direction of the Transparency indicating who is visible, and to whom \cite{heald_varieties_2006}. Transparency, according to Heald, has four directions on two planes: upwards and downwards; inwards and outwards. \textit{Upwards} Transparency indicates that those higher in a hierarchy can observe the conduct, behaviour, or results of those below -- whilst \textit{Downwards} Transparency is the means by which those lower in a hierarchy can observe those above them. In a similar dichotomy, \textit{Outwards} Transparency is used to describe how an organisation can see `outside', to either understand habitat or monitor peers and competitors. Situations where those external to an organisation can observe what's happening inside are described as examples of \textit{Inwards} Transparency \cite{heald_varieties_2006}. These can (and often do) coexist as any combination of the four i.e. having \textit{Upwards} Transparency does not preclude the existence of its counterpart, \textit{Downwards} Transparency, and when the two coexist there is a symmetry on that plane (vertical or horizontal) \cite{heald_varieties_2006}.

The combination of \textit{Downwards} and \textit{Outwards} Transparencies could be said to describe a form of surveillance, where employers can monitor their workers or a government can monitor the activities of its citizens \cite{heald_varieties_2006}. The inverse, \textit{Upwards} and \textit{Inwards}, implies a form of \textit{Sousveillance} -- a term coined by Steve Mann meaning \quoteit{to watch from below} \cite{mann_sousveillance:_2004}. Mann gives two possible interpretations of \textit{Sousveillance}, the first being the relative positions of the cameras in physical and social space (e.g. cameras on phones), with the other being an inversion of hierarchical surveillance and a close match to Heald's descriptions of \textit{Upwards} Transparency, where the citizen can peer into an organisation \cite{mann_sousveillance:_2004, heald_varieties_2006}. Sousveillance is often discussed in the context of citizens capturing abuses of power at street-level by trusted entities such as: police officers; security personnel; or similar entities, but Ganascia also discusses how an increased desire for access to public information has lead to aspirations for \quoteit{total transparency}, which in the US has resulted in governmental endorsement of data sharing (often called \textit{Open Government} or \textit{Government 2.0}) \cite{ganascia_generalized_2010}. There have been similar moves in the UK (e.g. data.gov.uk) which are designed to improve the delivery and accountability of public services \cite{shadbolt_linked_2012}.

\bollocks{Stuff re charity commission enabling certain transparencies.} An example of this from UK would be the presence of the Charity Commission, which enforces transparency measures that are applied to NPOs. Schauer writes of Transparency that for information, or processes, to qualify as transparent they must be \quoteit{open and available for scrutiny} \cite{schauer_transparency_2011}. This notably lacks an explanation of who can take advantage of transparent information, and at what cost to them. Transparency, according to Schauer, cannot be equated with knowledge; at best it facilitates it \cite{schauer_transparency_2011}. Hood similarly acknowledges this as an area of tension between the \quoteit{town meeting} form of Transparency, and its distant cousin concerned with accounting and book-keeping \cite{hood_transparency_2006-1}. Oliver echoes this sentiment, describing Transparency as a journey in contrast to a destination \cite{oliver_what_2004}.

Data released pertaining to resource allocation is an example of what Heald describes as \textit{Input Transparency}, which is easily measurable but doesn't necessarily address the link between this input and any outcome \cite{heald_fiscal_2003, heald_varieties_2006}. An NPO's nature as \textit{not-for-profit} additionally compounds the issues with using this data, since ``bottom-line'' measurements cannot be used, and allocation of funds to an activity or project doesn't indicate whether it is under, or over, funded \cite{henderson_performance_2002}. Oliver states that, where stakeholders are concerned, transparent communication involves \quoteit{focusing on more than just the traditional numbers, such as financial data, customer statistics, and operational metrics}, and that discussions of the stakeholders \quoteit{value drivers} should be had, and that they should be \quoteit{accurate and understandable} \cite{oliver_what_2004}. This forms part of a grander move, alluded to by Oliver and Schauer, away from \textit{Old Transparency} (Oliver) or \textit{Passive Transparency} (Schauer) and towards newer and more engaging forms of transparent practice.

Transparency in government and business has traditionally been a relatively passive affair, equating to the provision of data without particular concern to its promotion or interpretation \cite{schauer_transparency_2011, oliver_what_2004}. Even various `Freedom of Information' (FOI) Acts around the world, require that public companies provide information \textit{on request} but do not stipulate the format required -- and Worthy claims that whilst the UK Act has succeeded in making the government more transparent, it is not more trusted \cite{hood_transparency_2006-1, _freedom_2000, worthy_more_2010}. This attitude is summed up by Oliver as \quoteit{letting the truth be available to others to see if they so choose, or perhaps think to look, or have the time, means and skills to look} \cite{oliver_what_2004}. Societal factors such as the prevalence of information technology and media scrutiny have helped develop a movement towards more active forms of Transparency, called \textit{New Transparency} by Oliver and \textit{Active Transparency} by Schauer \cite{oliver_what_2004, schauer_transparency_2011}. These newer forms of Transparency are more concerned with the \quoteit{active disclosure} of information by organisations, seeing Transparency itself as an act of communication concerned with the organisation's responsibilities \cite{schauer_transparency_2011, oliver_what_2004}.

Modern information technologies play an important role in defining transparency, Meijer claims \quoteit{Modern transparency is computer-mediated transparency} \cite{meijer_understanding_2009}. A key characteristic in \quoteit{Computer-mediated Transparency} is that it shifts the direction of transparency from a two-way interaction such as in face-to-face interactions to a one-way interaction such as viewing data on a web page \cite{meijer_understanding_2009}. Meijer gives the example of citizens attending council meetings -- they can interrogate the councillors, but the councillors could also see citizens who had attended, whereas now councils cannot see who visits their web pages \cite{meijer_understanding_2009}.

%% :::::::::::::::::::::::::::::::::::::::::::::::::::::::::::::::::::::::::::::::::::::::::::::::::::::
\subsection{NPO Reporting Requirements and Responsibilities}
%% :::::::::::::::::::::::::::::::::::::::::::::::::::::::::::::::::::::::::::::::::::::::::::::::::::::
\bollocks{Discuss the principal-agent theory of accountability in npos}
%% ::::::::::::::::::::::::::::::::::::::::::::::::::::::::::::::::::::::::::
\subsection{Transparency and Accountability}
%% ::::::::::::::::::::::::::::::::::::::::::::::::::::::::::::::::::::::::::

% Introduce idea that transparency and accountability might not be linked
% Fox says that, aside from the muddle of transparency, there is also a muddle around accountability (define accountability)
% State question about different types of transparency generating different types of accountability
% Define the different types of transparency and accountability, make sure you discuss the focus on transparency goals (individual vs institutional)
% Make the point aht there is a difference between the official data, and the reliable RELEVANT information, leads to the fuzzy transparency , which means that an investment is 				required to transform data into transparent information.
% Introduce answerability, and then state that answerability is predicated on the ability to produce answers

The imposition of transparency measures is generally seen as tantamount to ensuring accountability of institutions, organisations, or individuals in power; and often the terms are used interchangeably \cite{fox_uncertain_2007, hood_accountability_2010}. The two terms, however, are separate and have their own (if somewhat malleable) definitions \cite{fox_uncertain_2007}.
%
Fox discusses accountability in terms of \quoteit{the capacity or right to demand answers} or the \quoteit{capacity to sanction}, whereas transparency concerns itself with the public's right and ability to access information; and whilst common wisdom dictates that transparency generates accountability, this assumption is challenged when held to scrutiny \cite{fox_uncertain_2007}. Fox's analysis of transparency divides it into two categories -- \textit{clear transparency} and \textit{opaque} or \textit{fuzzy transparency} -- which closely resemble Schauer and Oliver's definitions of the \textit{Active} or \textit{New Transparency} and the \textit{Passive} or \textit{Old Transparency} \cite{schauer_transparency_2011, oliver_what_2004, fox_uncertain_2007}. Fox argues the importance of this distinction lies in the fact that as Transparency becomes an increasingly desirable term, opponents will express their dissent through provision of \textit{fuzzy transparency}. This is data which lacks information that can reveal organisational behaviour and thus cannot be used to generate accountability \cite{fox_uncertain_2007}.
%
The \textit{Clear Transparency} alluded to by Fox is defined as \quoteit{information-access policies [and] programmes that reveal reliable information about institutional performance, specifying officials' responsibilities [and] where public funds go}. Importantly, though \textit{Clear Transparency} is concerned with organisational behaviour, it is not sufficient to generate accountability -- which requires the intervention of other actors \cite{fox_uncertain_2007}. Accountability is also explained by Fox as either \textit{Soft Accountability} (the ability to demand answers) and \textit{Hard Accountability} (the ability to issue sanctions). Fox stipulates that appropriate levels of \textit{Clear Transparency} gives the public the ability to perceive problems, and to demand answers -- which is a form of \textit{Soft Accountability} known as \textit{answerability} \cite{fox_uncertain_2007}. Further forms of accountability are founded on the ability to not only reveal existing data, but to investigate and produce information about organisational behaviour \cite{fox_uncertain_2007}.
%
Anheir and Hawkes reflect Fox's sentiment in their discussion of Accountability, where they describe Accountability as a \quoteit{multi-dimensional concept that needs unpacking before becoming a useful policy concept and management tool}, and note that in the case of trans-national organisations; Accountability itself is a problem, and not simply a solution \cite{anheier_accountability_2009}. This discussion, whilst focusing on the difficulty of regulating accountability across national borders, has insight into the ways that transparency mechanisms may not be adequate for generating true accountability in NPOs. They highlight how it is often media companies that reveal `unethical behaviour' to the public, rather than formal auditing bodies -- an example from the UK would be how the NPO \textit{Kid's Company} experienced negative media coverage over their closure relating to alleged misuse of funds \cite{elgot_kids_2015, anheier_accountability_2009}. Anheir and Hawkes also draw on Koppel's `Five Dimensions of Accountability' framework -- which imbues Accountability with a five-part typology: \textit{transparency}; \textit{liability}; \textit{controllability}; \textit{responsibility}; and \textit{responsiveness} \cite{anheier_accountability_2009, koppell_pathologies_2005}.
%
Koppel avoids trying to produce a definitive definition of Accountability, stating \quoteit{[to layer] every imagined meaning of accountability into a single definition would render the concept meaningless}, and the five-dimensional typology is instead designed to facilitate discussion of the term \cite{koppell_pathologies_2005}. Transparency features prominently in the typology, with Koppell referring to it as one of the \quoteit{foundations, supporting notions that underpin accountability in all of its manifestations} alongside liability \cite{koppell_pathologies_2005}. Liability, according to Koppell, is the attachment of consequences to performance and culpability to Transparency -- punishing organisations or individuals for failure, and rewarding them for successes \cite{koppell_pathologies_2005}. In this, the `foundational' dimensions of Koppell's typology are aligned with Fox's definitions of Accountability which covers the capacity of demanding answers and to sanction, with Transparency as the ability to access the information in the first place \cite{koppell_pathologies_2005, fox_uncertain_2007}.
%
The remaining three dimensions of Koppell's typology: \textit{controllability}; \textit{responsibility}; and \textit{responsiveness} are all built upon the foundations of Transparency and Liability. \textit{Controllability} is a form of accountability where if \quoteit{X can induce the behaviour of Y [then] X controls Y [and] Y is accountable to X} \cite{koppell_pathologies_2005}. Koppell notes that \textit{Controllability} can be difficult in organisations that have multiple stakeholders to whom the organisation is supposed to be controlled by \citep{koppell_pathologies_2005}. \textit{Responsibility} denotes the constraint of behaviour through laws, rules, or norms such as legal frameworks or professional standards of conduct. \textit{Responsiveness} in the typology describes the attention of an organisation to the needs of its clients, as opposed to the following of hierarchical orders \cite{koppell_pathologies_2005}.
%
\bollocks{MAD stuff here}.
%
\bollocks{if the charity commission is trying to ensure all aspects of accountability, is it subject to MAD?}
%
%% ===================================================================================================================================
\section{How can digital technologies support NPO Transparency and Accountability?}
%% ===================================================================================================================================

% Introducing this section.


% General things -- Open Data and Human Data Interactions
Transparency and Accountability can be said to be ultimately concerned with the sharing of information and the creation of pathways or mechanisms that allows stakeholders to act in accordance to it. Meijer argues that \quoteit{modern transparencies are computer-mediated} \cite{meijer_understanding_2009}, and Oliver goes as far to posit that modern technologies have sparked a self-sustaining \textit{Information - Transparency Cycle} which is \quoteit{unstoppable} and that information is now a commodity which is cheap to collect, organise, analyse, and distribute; the result of which is a reaction to missing information and a return to the collection phase \cite{oliver_what_2004}. Similarly, Steele, in his \textit{Open Source Everything Manfiesto} reflects on the ways in which The Internet has enabled the public to overcome previous restrictions on access to information and states, in no unclear terms, \quoteit{This bodes well for humanity} \cite{steele_open-source_2012}.

Broadly, there are several strands of research into digital technologies that support interacting with information and data in this way; primarily these can be encapsulated within the areas of Open Data, and Human-Data Interaction (HDI) although the inter-disciplinary nature of HCI as a whole means that the subject matter naturally intersects or otherwise touches upon other research within the field. This section explores the potential and implications for how data may be produced and appropriated by charities and their stakeholders through a discussion around Open Data, Open-Source Intelligence (OSINT) and Human-Data Interaction. It then turns to the pragmatic and investigates previous HCI research into previous examples pertaining to enabling interactions with finances


\textbf{This bit is orphaned at the moment! But intersesting to keep in mind}
An interesting case where Transparency has been built into the core of digital infrastucture is the Free Software movement and, to a lesser extent, the Open Source movement (for clarification please refer to (citation) ). In his original manifesto for Free Software, Stallman outlines the `Four Freedoms' that all Free Sofware grants, and how the licensing enforces these. These, in summary, are:

        The freedom to run the program as you wish, for any purpose (freedom 0).
        The freedom to study how the program works, and change it so it does your computing as you wish (freedom 1). Access to the source code is a precondition for this.
        The freedom to redistribute copies so you can help your neighbor (freedom 2).
        The freedom to distribute copies of your modified versions to others (freedom 3). By doing this you can give the whole community a chance to benefit from your changes. Access to the source code is a precondition for this.

Discussing this in terms of Transparency, Camp notes that the conditions for access to Source Code allows members of the community to effectively audit the code as one might with an organisation's accounts.

\section{Open Data, uses and implications}

% Open Data and its uses
Open Data is \quoteit{data that anyone can access, use, or share} \cite{open_data_institute_what_nodate}. It consists of organised data that is, generally, structured and placed online so that it can be consumed for use. Open Data can be produced, shared, and used by many people in many different contexts (e.g. scientific data sets, or government collection of environmental data). Often, it is parsed or processed in some way by digital technologies, and multiple datasets may also be combined in order to produce a desired insight for the stakeholder(s) consuming the data.

% Human Data Interaction and its conception of the use of data
Human Data Interaction (HDI) is an coalescing field that is concerned with the social world of how people interact with data about themselves and others. Whilst it considers technical infrastructure surrounding data \cite{mcauley_dataware_2011}, HDI also brings data's role as a `Boundary Object' \cite{star_institutional_1989} to the fore and considers its role as a pervasive aspect of everyday life in terms of how to enable citizens to interact with this data in a more explicit fashion \cite{mortier_human-data_2014}.

A `Boundary Object' is anything which may be recognised across different social `worlds', yet may be appropriated and adapted by the needs of individuals and groups in a manner that pertains to their specific needs and context. Star and Griesemer describe Boundary Objects as \quoteit{both plastic enough to adapt to local needs ... yet robust enough to maintain a common identity across sites} \cite{star_institutional_1989} A good example of this is a receipt of purchase; it is inarguably a receipt, yet may be used by the bank to verify a purchase, by a store to prove that you own the items you've purchased, and proof that a transaction has occurred between your bank account and a store. Further to this, Crabtree and Mortier elaborate that Boundary Objects are ``inherently social'' and possess a ``processual character'' as part of the infrastructure of everyday life. To this end, they argue that Data is not so much an object in-and-of-itself but rather an object that is inherently embedded in human relationships \cite{crabtree_human_2015}.

Data's use as a Boundary Object is demonstrated effectively by the rise in personal informatics. In the Quantified Self movement, individuals collect and process various forms and sources of data about them as individuals, generally for the purposes of recording progress towards various goals \cite{swan_emerging_2009, swan_quantified_2013}. Contrasting the movement's general use of data as a pragmatic, goal-oriented object, Elsden et al demonstrate that data can be experienced by people in many ways and can serve different purposes such as providing playful ways to engage with each other and one's own data \cite{elsden_quantified_2014, elsden_metadating:_2016}. In particular, it is posited that data can offer an `alternative lens' that other media does not, allowing people to view or represent an event in a different ways than originally envisioned and one that can be combined with other more traditional forms of documenting; and in doing so indicates that the record is always unfinalised and is continuously open to reinterpretation \cite{elsden_designing_2017}.

Additionally, data as a Boundary Object has place within an organisational context, offering opportunities to use and present data in an exploratory context in people's shared worlds. In an academic context (most UK Universities possess charitable status and thus belong to the Third Sector), visualising research funding across the institution was found to act as a means of supporting members of staff in understanding the funding landscape of the organisation and in communicating narratives to the outside world around perceived successes. The system (and by extension, its data) was also found to support the review and contestation of data when multiple interpretations were available, and Elsden et al explicitly note its implications for organisational transparency; with the caveat that a major design question raised by the research is whether contextualisation should occur merely through the data, or in conversation around it \cite{elsden_resviz:_2016}.

As discussed, the presentation of information is not enough to engage in more modern forms of transparent practice; and the use of data (however nicely it may be visualised) is no exception to this and risks simply rehashing the older forms of transparency with faster production of data. Cornford et al write that the UK Government's agenda of producing Open Government Data (OGD) fails to address the questions of how information is to be interpreted for local contexts; mirroring the concerns of Elsden et al around how data should be contextualised. Cornford et al argue that a wealth of open and structured data merely provides a `view from nowhere' and that the true challenge lies in developing the interpretive communities that will utilise the information effectively \cite{cornford_local_2013}.

Some successes have been seen in the uses of OGD for engagement; the London Datastore \cite{noauthor_london_nodate} collects, organises, and distributes a variety of Open Data pertaining to the City of London in the UK, and has seen great successes in terms of use-cases where developers and organisations have used data for service provision and governance purposes \cite{coleman_lessons_2013}. In the US, journalists have begun utilising data to fuel their practice \cite{ramos_journalists_2013}, and it has been argued that the proliferation of OGD has directly contributed to a `habit of engagement' that in turn begins to develop a culture of civic participation through the use and responses to data about our civic world \cite{gordon_making_2013-1}. Black and Burstein conceptualise a movement towards the `Twenty-First Century Town Hall' \cite{black_local_2013} (possibly indicating a full circle movement to a more direct forms of Transparency discussed above) whereas Bloom postulates that OGD will form a new Commons resource \cite{bloom_towards_2013}.

% More generally, hint towards OSINT practices and bring in the Steele stuff a bit more heavily.
Sense making and engagement stemming from the use of Open Data draws upon the field known as `Open-Source Intelligence' (OSINT). Generally, OSINT concerns itself with the gathering of intelligence for problem solving from various public information sources \cite{bradbury_plain_2011, glassman_intelligence_2012}. This places it in contrast to other forms of intelligence-gathering which are generally performed using specialist or secret sources of information. Traditionally, this would look like utilising sources such as newspapers and public records but in the modern era sources such as Open Data and Social Media profiles may be used as viable sources of information to begin making steps towards solving intelligence problems \cite{bizer_emerging_2009}.

% Link OSINT to transparent practice; if questions can be asked and answered effectively, have we achieved a more substantial form of transparency?
The implications of an Open Data-fuelled OSINT for Transparency and Accountability are evident. As discussed, OSINT is concerned primarily with the application of information to solve issues or answer questions, so it stands to reason that an adequate data infrastructure would allow for (or even promote) engagement of stakeholders in OSINT for asking questions of charitable organisations (the `Habit of Engagement alluded to earlier \cite{gordon_making_2013-1}). At the very least, such an infrastructure would enable the organisations to produce effective and interactive responses to queries around the performance of their work and their spending. 

Steele discusses the power of OSINT at length. Coming from a position that ``the only unlimited resource in the world is the human brain'', Steele puts forward that the engendering of Transparency via the production of Open Data, and the resultant OSINT practices, would lead to a systematic practice of exchanging information openly and allow societal actors and stakeholders to evaluate and respond to complex problems. Whilst Steele refers directly to engagement with Governmental processes, this would see use in the context of Charities as well when presenting information about their work within the context of complexity. A pragmatic example that might be most interesting for charities is the example of a `True Cost' calculation --  wherein the True Cost of a white cotton T-Shirt is outlined in economic, societal, and environmental terms.

In the context of charities, Erete et al explore how NPOs use Open Data technologies to support their practice through practices resembling OSINT. Data is largely used to create a narrative and engage in a story-telling practice around particular goals which differ in context -- e.g. making grant applications, or internal management functions. Organisations are shown to combine multiple sources of data into a narrative, as well as being able to derive multiple narratives from a single data set. Further to this, they discuss how NPOs operate with limited resources and as such may benefit from services such as Data Portals to enable them to acquire data easily to produce these valuable narratives, and put forward that additional value is created via such portals when they act to build or strengthen relationships between those seeking to use the data and those possessing skills or knowledge around its analysis. From the perspective of Transparency and Accountability, it stands to reason that systems can be developed that allows charities (NPOs) to engage actively in the data collection process, and allow them to construct multiple, and varied narratives from personal data sets that can be used in similar contexts to those described by Erete et al -- ie supporting grant applications and internal management procedures, but also additional cases such as evidencing their work by retrieving and presenting information collected about it.

In summation, it appears that digital technologies surrounding Open Data and its use as a Boundary Object have strong implications for the support of Charities in terms of Transparency and Accountability. Use of data has in the past demonstrated a usefulness in civic and academic contexts, supporting processes that are integral to Transparency (review and contestation), as well as potentially acting as a vector to allowing stakeholders to explore the complexities orbiting particular topics such True Cost, and therefore spending. With this in mind the discussion now turns to previous HCI work in the area of interacting with finances.


% ::::::::::::::::::::::::::::::::::::::::::::::::::::::::::::::::::::::::::::::::::::::::::::::::::::::::::::::::::::::::::::::::::::::::::::::::::::
\subsection{How do digital technologies allow people to interact with finances?}%
% ::::::::::::::::::::::::::::::::::::::::::::::::::::::::::::::::::::::::::::::::::::::::::::::::::::::::::::::::::::::::::::::::::::::::::::::::::::
HCI research has previously concerned itself with investigating the ways in which can facilitate people's interactions around money. Work has largely been focused on small scale interactions such as those found at individual or family/small-group level. Examples include studies investigating how people manage personal finances in particular circumstances, as well as how people engaged with money on an experiential level. At larger scales, HCI has also taken into account the social world around financial transactions to theorise around the design of potential future payment systems; and alternative forms of capital such as cryptocurrencies and the surrounding Blockchain technologies have accountability baked into the infrastructure of the systems themselves.
%

Previous HCI work demonstrates that people interact with money and their personal finances in a number of ways. Kaye et al discuss how interaction can be at an emotional level; as individuals may make decisions that do not appear 'rational' from a purely financial perspective but are instead driven by other factors such as personal history or experience with debt. A second facet is a form of management of 'pots' of money. In this context, money is not treated as a single entity but divided up along lines such as origin or intended use, and people use a variety of self-made or adaptable tools (both digital and analogue) in order to achieve this management; such as folders, notebooks, and spreadsheets. This is practice of dividing money semantically is notably also shown in the work of Vines et al when studying techniques people use to manage a low income. Finally, dealing with the unknown or 'higher powers' is an important facet of people's relationships with money as they may lack important information held about them; such as their current Credit Scores (often used as a measure of financial health in the US), and they understand that their personal futures may contain events that they have not financially planned or accounted for. Vines et al go on to describe how the systems people implement can give them a 'confidence through awareness' which may act as a ballast that partly allays their fears.


At the community Scale, Ferreira et al explored the social world surrounding money, specifically a community currency known as the Bristol Pound. Their work discusses how exchanges of money using the currency shared aspects of a conversation, as the transactors would engage in social interactions that were unbounded by the settings roles such as 'shopkeeper' and 'customer', prompted by the technology use required to pay with the currency. Further to this, the use of the shared currency (and the technological systems supporting it) gave the transactors an indication of shared values and interests.

Crowdfunding websites such as Kickstarter offer options wherein donators to a particular fund may have their donation returned if, for example, the total requested amount of donations has not been met. Beltran et al extend this concept further with their deployment of `Codo` which they describe as "Fundraising with Conditional Donations". In this deployment, Beltran et al describe how they developed a logical grammar which allows a donor on the system to more richly prescribe (or describe) the conditions of their donation, such as matching funds from other individuals or those within a defined group. Whilst this system does not proffer much in the way of exploring how organisations can report back on their expenditure, it presents the case that conditions may be put forward and codified as a means of providing a rudimentary accountability; as an organisation may need to attract the support of more than disparate groups in order to receive their donations. This opens up the possibility that a system may be developed with an 'accountability spin', where conditions are put upon funds by Funding bodies that request reimbursement under the event that conditions that they set out are not adequately met.

It is also important to note that my previous work has explicitly explored how Transparency and Accountability are, as of writing, poorly supported by digital technologies. This findings of this initial, exploratory, work highlights similarities between how individuals semantically divide money into 'pots' and how charities' finances are often restricted to particular use-cases due to how charity funding operates. This study also indicated that it may be appropriate to shift focus from financial transparency towards making organisation's 'visible'. This harks back to the historical roots of Transparency as a part-Science of making the social world knowable discussed earlier. The means to achieve this would be to produce a more qualitative form of accounting and supporting the interrogation of information collected by using standardised web technologies


% ===================================================================================================================================
\section{Opportunities}
% ===================================================================================================================================
% What is the financial and transparency work in NPOs?
% Lack in existing open data standards
% Lack in reporting processes
% Lack in the veracity of existing financial reporting for NPOS - i.e. transparency vs visibility
% Are these adequate?
% What can be done better?

% ===================================================================================================================================
\section{Summary}
\label{sec:related:conclusion}
% ===================================================================================================================================
