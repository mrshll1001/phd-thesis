\chapter{Methodology}
\label{sec:method}

\section{Introduction}
%This chapter discusses and interrogates the methodologies, approaches, and procedures used in the performance of the research project. It begins by describing the methodological framework that denoted how the project was organised, before turning to how data was collected and subsequently analysed. Finally, the chapter reflects on the mundane performance of the methods as they were affected during the production of the research, and the implications thereof.
%
%
%
%\section{Participation and Action Research}
%\label{sec:method:approach}
%
%Some intro stuff about action research.


\section{Interactions, Work, and Labour: Approaches to Fieldwork and Analysis}
\label{sec:method:fieldwork}

%The research initially set out to understand how digital systems can be used to do stuff with accounts and accountability in charities. 

%This section therefore seeks to justify the use of ethnomethodology as the orientation to the fieldwork performed in order to gather data, places the use of workshops within this, and also explores how a Marxist stance on political economy also influenced the analysis of fieldwork data and the subsequent design of systems developed as a response to it.


\subsection{Ethnomethodological Orientation to Fieldwork}

\subsection{Taking a Marxist stance}

The performance of the first large portion of fieldwork in the study, beginning March 2016, overlapped with a series of events in my personal life that began a pronounced interest in Marxism. This is worth mentioning because the process resulted in engagment with a host of classic, modern, and academic literatures.These had a subsequent effect on my reading of fieldwork data with regards to labour relations of workers, and helped shape the design of systems that were built in response.


% Introduction to Marxist theory as understood and used by the author. Focus on labour, reproductive labour, as means of analysing fieldwork data and why that is (+ shout out to acknowledge Marxist-Feminism)
Marxism, and therefore Marxist stances, can be said to be concerned primarily with the relationships between workers, their labour, capital, and property \bollocks{CITATIONS}. The foci offered by Marxism resonated with the research goals in as much as they were set out to determine in what ways the work of charities related to financial capital, and what role technology played in this relationship. Taking a Marxist stance meant, to me, orienting myself to how the labour of workers in charities was performed; how it was valued, and how it was represented in the various types of formal and informal accounts produced at the field site when accounting for activity or spending.

In Marxist theory, the work and value of any organisation, charities included, is ultimately derived from the labour of its workers \cite{marx_contribution_1970}. Classic political economy (e.g. \cite{smith_inquiry_1785}) divides labour into the categories of \textit{Productive Labour} and \textit{Unproductive Labour}. \textit{Productive Labour} is labour that directly goes into increasing the wealth of an organisation, whereas \textit{Unproductive Labour} is that which does not; but is nontheless necessary for the organisation or nation to function \cite{smith_inquiry_1785}. Marxism is famous for its critique of this classical and arbitrary divide of the productive and unproductive forms of labour; a critique that, in summary, attacks the notions of \textit{Productive} and \textit{Unproductive} as having no neutral definition; as work is not naturally productive and it takes work to make it so \cite{marx_capital_1974, marx_theories_1964}. Instead, what is seen as \textit{Productive} or \textit{Unproductive} is in constant flux and based on a multitude of different factors such as the mode of production\footnote{E.g. feudalism, capitalism, etc.}, social class, and what the `product' actually is \cite{marx_capital_1974}. When studying charities, this makes sense --- taking the classical approach, it would seem as if the only thing within a charity that could be called \textit{Productive Labour} is the act of writing applications for grants!

The Marxist stance on labour assisted in analysis of the fieldwork by providing a framework within which I could reason about the findings. Whilst ethnomethodology allowed for pragmatic insight into the interactional work of the field site itself, Marxian critiques of labour grounded these in what was needed to be captured to provide insight into design work. Namely, if the distinction between \textit{Productive} and \textit{Unproductive} Labour was based on an arbitrary dilineation, where was that line drawn in the field site? How did participants reason about what they classed as productive, and how was this accounted for to themselves and others? With Marxism being so closely related to my personal politics, I was initially apprehensive about taking this analytical stance. This anxiety was at least partially assuaged when I reflected upon findings of my previous study around charities and transparency. This study discussed that tensions around transparency in spending orbited around a focus on bottom line spending; leaving out the effort (labour) of volunteers and underplaying importance of the labour required of administrative tasks when discussing outcomes; therefore emphasising the importance of accounting for the nature of workers' labour in the field site and its relevance to design \cite{marshall_accountable:_2016}.

% and here's some stuff justifying using ethno alongside marxist critiques.
Further to this, the relationship between Marxism and Ethnomethodology has been explored. Chua explores Ethnomethodology from a Marxist perspective as means in which ideological reproduction may be investigated \cite{chua_delineating_1977}. Chua conceives of ideology as a symptom of social knowledge, and describes Ethnomethodology's focus on investigating the rational nature of a setting's activities as providing an analytical tool for Marxist sociology for investigating the how ideological systems operate as `natural' ways of knowing and interpreting the world \cite{chua_delineating_1977}. Chua finishes by discussing how the specificity of Ethnomethodologically-informed studies means that results must be appropriated within the wider context of Marxist sociology, but that Marxists should otherwise encourage Ethnomethogological studies \cite{chua_delineating_1977}. Conversely, investigating the relevance of Marx to Ethnomethodology, Mehan and Wood associate Ethnomethodology with a synthesis of both the hermeneutic-dialectial, and the logico-empiricist traditions; that whilst the methodology of Ethnomethodology is firmly rooted in logico-empricisim, its theory has derived from hermeneutic-dialectics, and Ethnomethodology's position as anathemic stems from its transcendancy of the two \cite{mehan_morality_1975}. Marxism similarly straddles these realms, simultaneously having philosophical roots in constructing Dialectical-Materialism whilst possessing what is widely regarded as a scientific-approach to the investigation of social life, and political economy \cite{thomas_marxism_2008}. Also of note is the similarity between Ethnomethodology's attendence to practical action \cite{garfinkel_studies_1967, crabtree_doing_2012} and Marx's thesis that \quoteit{All social life is essentially practical} \cite{marx_theses_1963} and indeed, Mehan and Wood argue that following the conception of Ethnomethodology, Marx can be viewed as a \quoteit{crypto-ethnomethodologist} \cite{mehan_morality_1975}.


% Then I want to talk about how marxist discussions on the modes of production influenced the design of the research, and how I justify that ethically.






% Marrying ethnomethodology and Marxist stances
% Discuss the papers doing just that, and add on the flavour of stuff re the Marxist position of what 'labour' is; ie if it's everything that we do, then surely the interactional work discussed as part of ethno is labour? And therefore justifiable as data from a marxist perspective.

% Finally, designing from a Marxist perspective. Anti-platform stance, design of federated systems.
% Justify WHY I chose to design in the federated aspect of the system when I was doing User-centred / participatory approaches -- backed up with fieldwork data plus some ethical stuff from Marx and Engels. Link back to anti-platform stance.


\section{Reflecting on the performance}

% Stuff on how long it took to actually build things?
