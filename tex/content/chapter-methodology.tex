\chapter{Methodology}
\label{sec:method}

\section{Introduction}
%This chapter discusses and interrogates the methodologies, approaches, and procedures used in the performance of the research project. It begins by describing the methodological framework that denoted how the project was organised, before turning to how data was collected and subsequently analysed. Finally, the chapter reflects on the mundane performance of the methods as they were affected during the production of the research, and the implications thereof.
%
%
%
%\section{Participation and Action Research}
%\label{sec:method:approach}
%
%Some intro stuff about action research.


\section{Interactions, Work, and Labour: Approaches to Fieldwork and Analysis}
\label{sec:method:fieldwork}

%The research initially set out to understand how digital systems can be used to do stuff with accounts and accountability in charities. 

%This section therefore seeks to justify the use of ethnomethodology as the orientation to the fieldwork performed in order to gather data, places the use of workshops within this, and also explores how a Marxist stance on political economy (and to a lesser extent, sociology) also influenced the analysis of fieldwork data and the subsequent design of systems developed as a response to it.


\subsection{Ethnomethodological Orientation to Fieldwork}

\subsection{Taking a Marxist stance}

The performance of the first large portion of fieldwork in the study, beginning March 2016, overlapped with a series of events in my personal life that began a slide across the political spectrum towards Communism. This is worth mentioning because the process resulted in engagment with a host of classic, modern, and academic literatures. These had a subsequent effect on my reading of fieldwork data, and helped shape the design of systems that were built in response.


% Introduction to Marxist theory as understood and used by the author. Focus on labour, reproductive labour, as means of analysing fieldwork data and why that is (+ shout out to acknowledge Marxist-Feminism)
Marxism, and therefore Marxist stances, can be said to be concerned primarily with the relationships between workers, their labour, capital, and property \bollocks{CITATIONS}. The foci offered by Marxism resonated with the research goals in as much as they were set out to determine in what ways the work of charities related to financial capital, and what role technology played in this relationship. In Marxist theory, the work and value of any organisation, charities included, is ultimately derived from the labour of its workers \cite{marx_contribution_1970}. Classic political economy (e.g. \cite{smith_inquiry_1785}) divides labour into the categories of \textit{Productive Labour} and \textit{Unproductive Labour}. \textit{Productive Labour} is labour that directly goes into increasing the wealth of an organisation, whereas \textit{Unproductive Labour} is that which does not; but is nontheless necessary for the organisation or nation to function \cite{smith_inquiry_1785}. Marxism is famous for its critique of this classical and arbitrary divide of the productive and unproductive forms of labour; a critique that, in summary, attacks the notions of \textit{Productive} and \textit{Unproductive} as having no neutral definition \cite{marx_capital_1974}. Instead, what is seen as \textit{Productive} or \textit{Unproductive} is in constant flux and based on a multitude of different factors such as the mode of production\footnote{E.g. feudalism, capitalism, etc.}, social class, and what the `product' actually is \cite{marx_capital_1974}.

%In this sense, then, the only thing that could be classed as Productive Labout within a charity would be the act of writing applications for grants


% I am trying to say that classical political economy divided labour into productive and unproductive, and that Marx also had some cool thoughts on it. This can be exemplified by my MRes study which was le published and whose findings begin to echo some of the labour of accounting I discuss elsewhere. This lead me to think that this was a valid way of analysing an appropriating fieldwork data, and here's some stuff justifying using ethno alongside marxist critiques.

% Then I want to talk about how marxist discussions on the modes of production influenced the design of the research, and how I justify that ethically.


%Preceding this research I had performed an exploratory study into charities which indicated that a key tension in representations of their finances, was the lack of data making clear the importance of both unpaid volunteer labour and the paid labour of administrative roles \cite{marshall_accountable:_2016}. This is highly reminiscent of theories of labour forms often held by Marxists, known as \textit{Productive} or \textit{Unproductive Labour} \cite{marx_capital_1974}.

\bollocks{Glue sentences}





% Marrying ethnomethodology and Marxist stances
% Discuss the papers doing just that, and add on the flavour of stuff re the Marxist position of what 'labour' is; ie if it's everything that we do, then surely the interactional work discussed as part of ethno is labour? And therefore justifiable as data from a marxist perspective.

% Finally, designing from a Marxist perspective. Anti-platform stance, design of federated systems.
% Justify WHY I chose to design in the federated aspect of the system when I was doing User-centred / participatory approaches -- backed up with fieldwork data plus some ethical stuff from Marx and Engels. Link back to anti-platform stance.


\section{Reflecting on the performance}

% Stuff on how long it took to actually build things?
